\section{DDL - Alter, Drop, Sequence}

\cm{INSERT}{Gegevens invoegen}
\cm{UPDATE}{Gegevens wijzigen}
\cm{DELETE}{Gegevens verwijderen}
\cm{TRUNCATE}{Tabellen leegmaken}
\cm{DROP}{Objecten verwijderen}
\cm{ALTER}{Objecten wijzigen}
\cm{ALTER ADD COLUMN|CONSTRAINT}{aan tabel toevoegen}
\cm{ALTER DROP COLUMN|CONSTRAINT}{van tabel verwijderen}
\cm{ALTER RENAME COLUMN|CONSTRAINT}{van tabel hernoemen}

\begin{tiny}
\begin{lstlisting}
  INSERT INTO foo (foo, bar) VALUES (a, b);
\end{lstlisting}

\begin{lstlisting}
  UPDATE tabelnaam
  SET attribuutnaam=nieuwe waarde
  WHERE conditie;
\end{lstlisting}

\begin{lstlisting}
  DELETE FROM deps USING emp
  WHERE deps.emp_id=emp.id
  AND lower(name)='bob';
\end{lstlisting}
\end{tiny}

\subsection{Sequence}
\say{Wordt gebruikt om automatisch volgnummers te genereren.}

\begin{tiny}
\begin{lstlisting}
  CREATE SEQUENCE [IF NOT EXISTS] sequencenaam
  INCREMENT [BY ] increment
  [MINVALUE minvalue | NO MINVALUE]
  [MAXVALUE maxvalue| NO MAXVALUE]
  [START [WITH] startwaarde
  [CACHE cache]
  [[NO] CYCLE]
\end{lstlisting}

\begin{lstlisting}
  CREATE SEQUENCE seq_project_id
  START WITH 100 INCREMENT BY 5;
  -- 100 105 110 115 120...

  INSERT INTO projects
  VALUES ( nextval('seq_project_id'), 'Foo Bar');
\end{lstlisting}
\end{tiny}
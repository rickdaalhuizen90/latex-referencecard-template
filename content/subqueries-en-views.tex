\section{Subqueries}
Een subquery is een query in een andere query.
\\
• Wordt eerst de subquery uitgevoerd.\\
• Levert waarden aan de hoofdquery.\\

\begin{tiny}
\begin{lstlisting}
  SELECT *
  FROM emplyoees
  WHERE salary = (
    SELECT MIN(salary) FROM employees
  );
\end{lstlisting}
\end{tiny}
\say{Eerst wordt de binnenste $SELECT$ uitgevoerd,\\
daarna de buitenste
}
\\\\
Waar kan je subquery schrijven?\\
$SELECT, FROM, WHERE, HAVING$
\\\\
Subqueries die meer dan 1 rij opleveren\\
\cm{$<ANY$}{Minder dan het hoogste}
\cm{$>ANY$}{Meer dan de laagste}
\cm{$=ANY$}{gelijk aan één van de resultaten}
\cm{$!=ANY$}{verschillend van één van de resultaten}
\cm{$>ALL$}{Meer dan de hoogste}
\cm{$<ALL$}{Minder dan de laagste}
\cm{$<>ALL$}{verschillend van alle resultaten}
\\
Alternatief:
\cm{$<MAX$}{Minder dan het hoogste}
\cm{$>MIN$}{Meer dan de laagste}
\cm{$IN$}{gelijk aan één van de resultaten}
\cm{$>MAX$}{Meer dan de hoogste}
\cm{$<MIN$}{Minder dan de laagste}
\cm{$NOT IN$}{verschillend van alle resultaten}

\subsection{Views}
Via een view kan men gegevens van de\\
onderliggende tabel afschermen.

\begin{tiny}
\begin{lstlisting}
  CREATE OR REPLACE VIEW v_dept_view
    AS SELECT department_id,department_name
      FROM DEPARTMENTS
      ORDER BY department_id;
\end{lstlisting}
\end{tiny}
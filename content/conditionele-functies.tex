\section{Condtionele functies}

\begin{tiny}
\begin{lstlisting}
  SELECT employee\_id, manager\_id,
  GREATEST(employee\_id, manager\_id),
  LEAST(employee\_id, manager\_id)
  FROM employees;
\end{lstlisting}

\begin{lstlisting}
  SELECT employee_id, province,
    CASE province
      WHEN 'NB' THEN 'Noord Brabant'
      WHEN 'LI' THEN 'Limburg'
      ELSE province
    END "Full name"
  FROM employees;
\end{lstlisting}
\end{tiny}

\subsection{COALSCE}
• De functie geeft de eerste $NOT NULL$ parameter\\
van de parameterlijst terug.\\
• De $COALESCE$ functie heeft minstens 2 parameters\\
• Wanneer alle parameters een $NULL$ waarde bevatten,\\
geeft de functie $NULL$ terug.\\

\begin{tiny}
\begin{lstlisting}
  SELECT COALESCE(hours, 0) FROM tasks;
\end{lstlisting}

\begin{lstlisting}
  SELECT name,
    COALESCE(email, phone, cellphone) contact
  FROM contact_info;
\end{lstlisting}
\end{tiny}

\subsection{NULLIF}
• Geeft NULL terug als beide expressies hetzelfde\\
resultaat geven.\\
• Bij ongelijkheid wordt de eerste parameter teruggegeven.

\begin{tiny}
\begin{lstlisting}
  NULLIF(value1, value2)
\end{lstlisting}
\end{tiny}